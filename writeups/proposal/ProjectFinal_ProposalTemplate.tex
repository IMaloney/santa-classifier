%%%%%%%%%%%%%%%%%%%%%%%%%%%%%%%%%%%%%%%%%%%%%%%%%%%%%%%%%%%%%%%%%%%%%%%%%%%%%%%%%%%%%%%%%%%%%%%%
%
% CSCI 1430 Written Question Template
%
% This is a LaTeX document. LaTeX is a markup language for producing documents.
% Your task is to answer the questions by filling out this document, then to
% compile this into a PDF document.
%
% TO COMPILE:
% > pdflatex thisfile.tex

% If you do not have LaTeX, your options are:
% - VSCode extension: https://marketplace.visualstudio.com/items?itemName=James-Yu.latex-workshop
% - Online Tool: https://www.overleaf.com/ - most LaTeX packages are pre-installed here (e.g., \usepackage{}).
% - Personal laptops (all common OS): http://www.latex-project.org/get/ 
%
% If you need help with LaTeX, please come to office hours.
% Or, there is plenty of help online:
% https://en.wikibooks.org/wiki/LaTeX
%
% Good luck!
% The CSCI 1430 staff
%
%%%%%%%%%%%%%%%%%%%%%%%%%%%%%%%%%%%%%%%%%%%%%%%%%%%%%%%%%%%%%%%%%%%%%%%%%%%%%%%%%%%%%%%%%%%%%%%%
%
% How to include two graphics on the same line:
% 
% \includegraphics[width=0.49\linewidth]{yourgraphic1.png}
% \includegraphics[width=0.49\linewidth]{yourgraphic2.png}
%
% How to include equations:
%
% \begin{equation}
% y = mx+c
% \end{equation}
% 
%%%%%%%%%%%%%%%%%%%%%%%%%%%%%%%%%%%%%%%%%%%%%%%%%%%%%%%%%%%%%%%%%%%%%%%%%%%%%%%%%%%%%%%%%%%%%%%%

\documentclass[11pt]{article}

\usepackage[english]{babel}
\usepackage[utf8]{inputenc}
\usepackage[colorlinks = true,
            linkcolor = blue,
            urlcolor  = blue]{hyperref}
\usepackage[a4paper,margin=1.5in]{geometry}
\usepackage{stackengine,graphicx}
\usepackage{fancyhdr}
\setlength{\headheight}{15pt}
\usepackage{microtype}
\usepackage{times}
\usepackage{booktabs}

% From https://ctan.org/pkg/matlab-prettifier
\usepackage[numbered,framed]{matlab-prettifier}

\frenchspacing
\setlength{\parindent}{0cm} % Default is 15pt.
\setlength{\parskip}{0.3cm plus1mm minus1mm}

\pagestyle{fancy}
\fancyhf{}
\lhead{Final Project Proposal}
\rhead{CSCI 1430}
\rfoot{\thepage}

\date{}

\title{\vspace{-1cm}Final Project Proposal}

\begin{document}
\maketitle
\vspace{-1cm}
\thispagestyle{fancy}

\textbf{Team name: \emph{SANTA IS REAL}}

% \begin{itemize}
%   \item What is your project idea?  
%   \item What is the socio-historical context that this project lives in? 
%   \item Please list three stakeholders that your project could impact, and describe how it could impact them.
%   \item -----
%   \item What are the skills of the team members? Conduct a skill assessment!
%   \item What data will you use?
%   \item What software/hardware will you use?
%   \item Who will do what? [For anonymity, please use `'`Team member 1 will...'' or, alternatively, take on daring pseudonames.]
%   \item How will you know whether you have made progress? What will you measure?
%   \item What technical problems do you foresee or have?
%   \item Is there anything that we can do to help? E.G., resources, equipment.
% \end{itemize}

\section*{Introduction}
Have you ever found yourself in a situation where you weren't sure the jolly, bearded figure standing in front of you on the train
is the Santa Claus or just a poser? What if you could put those doubts to rest with just 
a simple picture? As we all know, it is of the upmost importance to be able to recognize Santa at all times, for he is 
the harbinger of holiday cheer, joy, love and life. With that in mind, we introduce "Is that Santa?" – a model designed to identify 
Santa from a picture. The model will be demonstrated with an iphone application can take a picture of the person, then determine 
the likelihood that person is Santa.  

\section*{Socio-Historical Context}
This project finds its roots in the evolving relationship between technology and cultural traditions. The idea of Santa Claus has 
existed around the world for centuries. Yet, Santa is not a stagnant figure. Over the years, he has been protrayed in many 
different lights. In the United States more specifically, Santa has become a more diverse figure as he is not defined by his 
race, but more his common features, such as red attire, white beard, rosy cheeks, etc. This project hopes to rectify some of the differences 
in the protrayal of Santa across the United States. Taking it a step further, "Is That Santa" exists within a larger trend of leveraging advanced 
technology to analyze and interpret cultural phenomena. In this context, the development of a model that 
can accurately classify whether a person is Santa Claus or not is a natural extension of the ongoing exploration of 
technology's potential to enhance our understanding of the world around us.
% By focusing on a lighthearted subject 
% like Santa Claus, this project showcases the versatility of computer vision and deep learning while also emphasizing the potential 
% for these technologies to enrich our cultural experiences.

\section*{Data}
There is a dataset on Kaggle comprised of \href{https://www.kaggle.com/datasets/deepcontractor/is-that-santa-image-classification/code}{Santa images} to train the model.
{images of Santa}. This dataset, however, is not fully representative of how Santa Claus is in America. We intend on building our own dataset to have a more representative 
and robust group of images. To do so, we plan on writing a simple script that scraps the web for Santa images. Afterwards, we plan to do a quick look through each image to see 
how good it is of representing Santa. The goal is to train on 10 Gbs worh of images, however, there may be some limitations with resources based on this amount of data.


\section*{Team}
The team's current members and skill sets are as follows:
\begin{enumerate}
  \item Member 1:
  \begin{itemize} 
    \item \textbf{About}: full time 
    systems programmer working in Cambridge, Massachusetts
    \item \textbf{Skills}: Android, iOS, distributed systems, building data pipelines, AWS/GCP
  \end{itemize}
\end{enumerate}
\section*{Hardware and Software}
\subsection*{Hardware}
GCP offers many different types of GPUs in order to train your model. For this project, the decision fell in between 
the NVDIA Tesla T4 (the GPU used for colab instances) and the next tier GPU, the NVIDIA Tesla P100. The choice was between these 
two GPUs because they are the cheapest on the platform. We will use the T4 for this project for a few reasons:
\begin{enumerate}
  \item \textbf{Price}: The T4 is cheaper at \$0.95 per hour versus the P100's \$1.46 per hour.
  \item \textbf{Tensor cores} : The T4 has tensor cores designed for deep learning tasks. These cores speed up multiplication and, more importantly, 
        convolution, which will be a very important step towards classification. 
\end{enumerate} 

\subsection*{Software}
This is a tentative list (most more than likely, software will be added):
\begin{enumerate}
  \item \textbf{Webscraping Script:}
  \begin{itemize}
    \item BeautifulSoup -- parsing html
    \item urllib.request -- making requests
    \item robotexclusionrulesparser -- parsing robots.txt file
  \end{itemize}
  \item \textbf{Model:}
  \begin{itemize}
    \item tensorflow -- developing model
    \item skimage -- image manipulation and processing 
  \end{itemize}
  \item \textbf{Demo App:}
  \begin{itemize}
    \item tensorflow lite -- evaluating model on iphone
  \end{itemize}
\end{enumerate}

\section*{Road Map}
The following progress points for the project exists:
\begin{enumerate}
  \item Data Collection (estimated: 2 Days)
        \begin{itemize}
          \item Writing script to pull data from web in a way complaint with Google terms and services
          \begin{enumerate}
            \item use Google query APIs (rate limited to 100 queries a day)
            \item abide by robots.txt files for sites in consideration
          \end{enumerate}
        \end{itemize}
  \item Setting up cloud resources (estimated: 1-2 Days)
  \begin{itemize}
    \item Creating a project in GCP
    \item Setting up a storage bucket
    \item Basic processing and formatting of data
    \item Set up an AI Platform training job
  \end{itemize}
  \item Creating and training model (estimated: 17 Days)
  \begin{itemize}
    \item Defining layers
    \item Standardization and Data Augmentation
    \item Adjusting hyperparameters
  \end{itemize}
  \item Developing demo app (estimated: 5 Days)
  \item Preparing Presentation (estimated: 2 Days)
\end{enumerate}

\section*{Possible Technical Problems}
The two potential issues revolve around data collection and effficient resource usage.
\begin{enumerate}
  \item \textbf{data collection:} Webscraping Google is subject to its fair terms and use guidance. Depending on how the images are scraped this could present a breach of the terms of 
  service, which would be bad. To circumvent this, we would either have to be very careful with respecting Google's rules regarding bots (such as obeying the robots.txt file) or use the Google query api, which is 
  extremely rate limited. If we cannot sucessfully webscrape Google images, we would have no choice but to use their apis which would make data collection take a very long time.
  \item \textbf{resource usage:} If we are not granted GCP credit to run the model, we would be forced to use colab or the grid. Neither of these options are ideal as they are either more confusing to use or they are 
  bound by other major limitations (such as allowed time to run, memory, etc.).
\end{enumerate}
\section*{How Can CS1430 Help?}
CS1430 could help our team out the most by providing the resources necessary to spin up the training necessary. If it would be impossible to get some GCP credit, we would have to change how we 
train the model. One consideration is a similar approach to homework 5, where we would save checkpoints and use each checkpoint to further train after a time out. This would make Google colab a viable 
option as it uses the same GPU discussed in the hardware section that we planned on using. However, using colab will make it more difficult to run as colab has its own fixed memory which may not be large 
enough to run both the program and store the entire dataset.

\end{document}